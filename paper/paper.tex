\documentclass[letterpaper]{article}
% DO NOT CHANGE THIS
\usepackage{aaai24} % DO NOT CHANGE THIS
\usepackage{times} % DO NOT CHANGE THIS
\usepackage{helvet} % DO NOT CHANGE THIS
\usepackage{courier} % DO NOT CHANGE THIS
\usepackage[hyphens]{url} % DO NOT CHANGE THIS
\usepackage{graphicx} % DO NOT CHANGE THIS
\urlstyle{rm} % DO NOT CHANGE THIS
\def\UrlFont{\rm} % DO NOT CHANGE THIS
\usepackage{graphicx}  % DO NOT CHANGE THIS
\usepackage{natbib}  % DO NOT CHANGE THIS
\usepackage{caption}  % DO NOT CHANGE THIS
\frenchspacing % DO NOT CHANGE THIS
\setlength{\pdfpagewidth}{8.5in} % DO NOT CHANGE THIS
\setlength{\pdfpageheight}{11in} % DO NOT CHANGE THIS
%
% Keep the \pdfinfo as shown here. There's no need
% for you to add the /Title and /Author tags.
\pdfinfo{
/TemplateVersion (2024.1)
}

\setcounter{secnumdepth}{0} %May be changed to 1 or 2 if section numbers are desired.

\title{Shepherd: An Incremental Story Sifting-Based Drama Manager}
\author {
    Sage Deo,
    Jonathan Chung,
    Joshua McCoy
}
\affiliations {
    University of California, Davis \\
    \{rsdeo, secondAuthor, jamccoy\}@ucdavis.edu
}

\begin{document}
\maketitle

\begin{abstract}
    Test abstract.
\end{abstract}

\section{Introduction}
In the field of interactive emergent narrative, there is a distinction made between a
chronicle of simulated events and a narrative. The key element that transforms a chronicle
into a narrative is curation, the process of cutting out narratively irrelevant content
from a chronicle to create a narrative. However, this raises an important question: how do
we decide which parts of a chronicle constitute a narrative?

Oftentimes, as in the case of Bad News, humans are able to intuitively find small
sequences of narratively interesting content, and then assemble those sequences into a
larger narrative. This process has been termed “story sifting.” To aid in automatic
narrative curation, automatic story sifters such as Felt \cite{kreminski:felt} and Sheldon
\cite{ryan:curating} were created. These
automatic story sifters accept human-authored “story templates,” which are high-level
descriptions of small sequences of narratively interesting material. Given a chronicle of
simulated events and a set of story templates, automatic story sifters can identify themes
such as “violation of hospitality” or “repeated betrayals,” which can later be assembled
into a narrative. 

Later, the notion of “incremental story sifting” was implemented in Winnow
\cite{kreminski:winnow}, the successor
language to Felt. Winnow is capable of identifying partially-fulfilled story templates in
a simulation currently in progress, which the authors note could potentially be used for
foreshadowing future plot developments, i.e. the completion of the partially-filled
template in question. However, this also opens up the possibility of using Winnow, or any
other incremental story sifter, as a sort of drama manager to guide a simulation in
progress towards making choices that progress any partially-filled story templates it
detects. This was explored to some degree by Loose Ends \cite{kreminski:loose}, a mixed-initiative authoring
system which uses Winnow to suggest plot developments for a human to choose between.
However, this technique has not yet been explored as part of a fully autonomous
simulation, without any human guidance. 

In this work, we contribute the first (to our knowledge) instance of a story sifter being
leveraged for drama management in a simulation with no human input, as well as an
evaluation of the stories such a system can construct. Our system differs from other story
sifters due to the fact that it sifts alongside the generator, allowing our system to have
focused, curated narratives alongside the variance of a chaotic generator. It nudges the
course of the narrative rather than outright demanding that certain events take place,
allowing our system to follow story structures in interesting ways while still retaining a
degree of control and coherence. The creativity and cohesion of the stories our system
will generate will be proof that this method of incremental story sifting alongside
generation is a viable and ripe path to pursue.

\section{Related Work}
This work is very informed by Curating Simulated Storyworlds \cite{ryan:curating} and its
philosophy on story sifting and emergent narrative, especially as that philosophy is
instantiated in the theatrical performance piece Bad News. We also directly use the Winnow
incremental story sifter in our implementation of this system. Our
use of story sifting as a drama manager, to guide a simulation in progress, is a
simulationist extension of its use in the Loose Ends system. 
% TODO: a versu citation should go here
Where
Loose Ends is a “strong story” system, in which a story sifter acts to guide a human
author with ultimate control over the plot, our proposed system takes a “strong autonomy”
approach, where the story sifter guides the characters, who otherwise make their own
decisions. 

Guided by the work of Max Kreminski, Noah Wardrip-Fruin, and Michael Mateas in Authoring
For Story Sifters \cite{kreminski:authoring}, we intend to address the most glaring issue
with interactive emergent narrative: “the dissolution of the player-perceived story into a
structureless mess”. While Shepherd characters make their own decisions, they are guided
by the drama manager to advance and complete story arcs. By sifting simultaneously with
generation, Shepherd provides structure to the story itself rather than looking for
narratives post-generation as the wizard of Bad News does. Although this sacrifices some
randomness and chaos during the generation phase, the corresponding reward of coherence is
far more impactful. In addition, because Shepherd remains a “strong autonomy” system, the
characters continue to make decisions and create stories that surprise. As a result,
Shepherd will generate stories that deviate from the established story templates which
players will narrativize themselves as they do in Dwarf Fortress or The Sims.
% TODO: cite DF/Sims?

\section{Technical Description}
The basic building blocks of a Shepherd world are time, characters, actions, traits, and
templates. The interactions between all of these pieces, as well as the drama manager,
form the basis for rich emergent stories.

\subsection{Time}
In Shepherd, time is divided into a series of discrete time steps. For each time step,
each character in the simulation may perform one action. For ease of computation, all
actions are internally represented as taking place in sequence. However, readers may
construct narratives in which actions within the same time step happen in parallel. This
openness to interpretation helps facilitate narrativization.

\subsection{Characters}
Characters in Shepherd are semi-autonomous agents. Every time step, each character
generates a utility score for each possible action they can take, based on that
character's traits. The utility score is combined with the drama manager's score for that
action to produce an overall score. Then, the character makes a weighted random choice
between all possible actions, with each action's overall score used as its weight.

This method of modeling character decisions allows for character choices to follow
consistent trends, which is good for allowing the audience to understand each character.
However, it also allows characters to infrequently act out of character, creating
narrative intrigue.

\subsection{Actions}
An action is an event in the story world that can be triggered by some character (the
actor). Some actions are monadic, and have only an actor; however, other actions are
dyadic, and involve a separate character who is not the actor (the target). Each character
may only serve as the actor for one action per time step, but each character can serve as
the target for arbitrarily many actions per time step.

Each action also has a set of tags that provide higher-level information about each event:
some examples of tags are ``friendly,'' ``food,'' and ``violent.'' Tags mainly interact
with the rest of the system in two ways. First, when evaluating an action (via their
traits), characters consider that action's tags when determining its utility score.
Second, human-authored story templates can match actions based on tags. This allows a
single line in a pattern to represent a wide variety of actions which are all narratively
equivalent in the context of that pattern.

\subsection{Traits}
Traits are the main agents of character personality modeling, and provide the basis for
utility-based character decision-making. Each character is assigned a consistent number of
(currently, two) traits at character generation, and each trait has its own utility
function. A character's utility score for a given action is the sum of the utility scores
for each of the character's traits evaluated on that action. At a high level, this means
that each character is likely to act in ways that are consistent with each of their
traits: for example, a character who is friendly (high utility for actions tagged
``friendly'') and a gourmet (high utility for actions tagged ``food'') is very likely to want to
share a meal with some other character, because sharing a meal is tagged as both
``friendly'' and as ``food.'' 

Like tags, traits can also be used as conditions when writing templates. Thus, one
character's traits may influence other characters' behavior via the drama manager.

\subsection{Templates and the Drama Manager}
Each template represents a sequence of narratively interesting events, such as a ``drunken
brawl'' or ``failed flirtation.'' Templates are expressed as patterns in the Winnow
language. Shepherd maintains a list of possible, partially filled, and complete templates;
this list is updated each time an action is taken. Furthermore, when the drama manager
evaluates an action, it checks to see if that action would advance any of its tracked
story templates. Actions that advance many story templates are valued more highly by the
drama manager, as are actions that advance story templates which are close to completion.
This makes it more likely that characters take actions that are narratively interesting
and which work towards resolving open plot threads.

% TODO: add more?

%We propose a very simple simulation. Each simulation is instantiated with a set of
%characters, a set of traits for those characters, and a set of actions for those
%characters to take. Actions may apply to nothing, or may apply to some other character in
%the simulation. The simulation handles time as a series of time-steps, where each
%character may take one action per time-step in a predetermined order.  
%
%For each character on each time-step, the character will choose an action according to the
%following procedure:
%
%\begin{itemize}
%    \item Generate a pool of possible actions for that character, based on
%        \begin{itemize}
%            \item Actions that advance existing story goals
%            \item A few randomly generated actions, to add unpredictability
%            \item A pool of always-available actions
%            \item A pool of conditionally-available actions whose preconditions are met
%        \end{itemize}
%    \item 
%        Generate a utility score for each action. Utility is based on
%        \begin{itemize}
%            \item The character’s traits
%            \item The relationship between the character and the target of the action (if
%                applicable)
%            \item How many story goals the action would progress
%        \end{itemize}
%    \item Pick an action from the pool randomly, where actions with higher utility scores
%        are more likely to be chosen 
%    \item Update the simulation state to account for the effects of the chosen action
%\end{itemize}
%
%Each action will have an associated natural-language representation template, which will
%be filled in with the appropriate character names once the action is performed. This
%natural-language string will then be written to the frontend to expose that event to the
%reader as part of the simulation chronicle. 

\section{Analysis}
TBD

\section{Conclusion}
TBD

\subsection{Future Work}
In the future, we plan to expand the system by
\begin{itemize}
    \item using it to implement a murder-mystery game which will be called
        \textit{Assassin: The Masquerade}.
    \item adding text to each performed action that contextualizes the action with respect to each
        story template step that is matched by that action
    \item adding graphics that visually depict each action as it is performed
    \item changing the UI to make it easier to track each individual character's story
    \item possibly adding the option to filter the chronicle to only show actions
        involving some character (i.e. show only actions where the chosen character is the
        target or actor).
\end{itemize}

\bibliography{refs}

\end{document}
