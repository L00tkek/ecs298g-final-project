\documentclass[letterpaper]{article}
% DO NOT CHANGE THIS
\usepackage{aaai24} % DO NOT CHANGE THIS
\usepackage{times} % DO NOT CHANGE THIS
\usepackage{helvet} % DO NOT CHANGE THIS
\usepackage{courier} % DO NOT CHANGE THIS
\usepackage[hyphens]{url} % DO NOT CHANGE THIS
\usepackage{graphicx} % DO NOT CHANGE THIS
\urlstyle{rm} % DO NOT CHANGE THIS
\def\UrlFont{\rm} % DO NOT CHANGE THIS
\usepackage{graphicx}  % DO NOT CHANGE THIS
\usepackage{natbib}  % DO NOT CHANGE THIS
\usepackage{caption}  % DO NOT CHANGE THIS
\frenchspacing % DO NOT CHANGE THIS
\setlength{\pdfpagewidth}{8.5in} % DO NOT CHANGE THIS
\setlength{\pdfpageheight}{11in} % DO NOT CHANGE THIS
%
% Keep the \pdfinfo as shown here. There's no need
% for you to add the /Title and /Author tags.
\pdfinfo{
/TemplateVersion (2024.1)
}

\setcounter{secnumdepth}{0} %May be changed to 1 or 2 if section numbers are desired.

\title{Shepherd: An Incremental Story Sifting-Based Drama Manager}
\author {
    % Authors
    First Author Name,
    Jonathan Chung,
    Joshua McCoy
}
\affiliations {
    % Affiliations
    University of California, Davis \\
    \{firstAuthor, secondAuthor, jamccoy\}@ucdavis.edu
}

\begin{document}
\maketitle

\begin{abstract}
    Test abstract.
\end{abstract}

\section{Introduction}
In the field of interactive emergent narrative, there is a distinction made between a
chronicle of simulated events and a narrative. The key element that transforms a chronicle
into a narrative is curation, the process of cutting out narratively irrelevant content
from a chronicle to create a narrative. However, this raises an important question: how do
we decide which parts of a chronicle constitute a narrative?

Oftentimes, as in the case of Bad News, humans are able to intuitively find small
sequences of narratively interesting content, and then assemble those sequences into a
larger narrative. This process has been termed “story sifting.” To aid in automatic
narrative curation, automatic story sifters such as Felt and Sheldon were created. These
automatic story sifters accept human-authored “story templates,” which are high-level
descriptions of small sequences of narratively interesting material. Given a chronicle of
simulated events and a set of story templates, automatic story sifters can identify themes
such as “violation of hospitality” or “repeated betrayals,” which can later be assembled
into a narrative. 

Later, the notion of “incremental story sifting” was implemented in Winnow, the successor
language to Felt. Winnow is capable of identifying partially-fulfilled story templates in
a simulation currently in progress, which the authors note could potentially be used for
foreshadowing future plot developments, i.e. the completion of the partially-filled
template in question. However, this also opens up the possibility of using Winnow, or any
other incremental story sifter, as a sort of drama manager to guide a simulation in
progress towards making choices that progress any partially-filled story templates it
detects. This was explored to some degree by Loose Ends, a mixed-initiative authoring
system which uses Winnow to suggest plot developments for a human to choose between.
However, this technique has not yet been explored as part of a fully autonomous
simulation, without any human guidance. 

In this work, we contribute the first (to our knowledge) instance of a story sifter being
leveraged for drama management in a simulation with no human input, as well as an
evaluation of the stories such a system can construct. Our system differs from other story
sifters due to the fact that it sifts alongside the generator, allowing our system to have
focused, curated narratives alongside the variance of a chaotic generator. It nudges the
course of the narrative rather than outright demanding that certain events take place,
allowing our system to follow story structures in interesting ways while still retaining a
degree of control and coherence. The creativity and cohesion of the stories our system
will generate will be proof that this method of incremental story sifting alongside
generation is a viable and ripe path to pursue.

\section{Related Works}
This work is very informed by Curating Simulated Storyworlds \cite{ryan:curating} and its
philosophy on story sifting and emergent narrative, especially as that philosophy is
instantiated in the theatrical performance piece Bad News. We also directly use the Winnow
incremental story sifter \cite{kreminski:winnow} in our implementation of this system. Our
use of story sifting as a drama manager, to guide a simulation in progress, is a
simulationist extension of its use in the Loose Ends system \cite{kreminski:loose}. Where
Loose Ends is a “strong story” system, in which a story sifter acts to guide a human
author with ultimate control over the plot, our proposed system takes a “strong autonomy”
approach, where the story sifter guides the characters, who otherwise make their own
decisions. 

Guided by the work of Max Kreminski, Noah Wardrip-Fruin, and Michael Mateas in Authoring
For Story Sifters \cite{kreminski:authoring}, we intend to address the most glaring issue
with interactive emergent narrative: “the dissolution of the player-perceived story into a
structureless mess”.

\section{Technical Description}
We propose a very simple simulation. Each simulation is instantiated with a set of
characters, a set of traits for those characters, and a set of actions for those
characters to take. Actions may apply to nothing, or may apply to some other character in
the simulation. The simulation handles time as a series of time-steps, where each
character may take one action per time-step in a predetermined order. Furthermore, the
simulation tracks a friendliness value between each pair of characters.  

For each character on each time-step, the character will choose an action according to the
following procedure:

\begin{itemize}
    \item Generate a pool of possible actions for that character, based on
        \begin{itemize}
            \item Actions that advance existing story goals
            \item A few randomly generated actions, to add unpredictability
            \item A pool of always-available actions
            \item A pool of conditionally-available actions whose preconditions are met
        \end{itemize}
    \item 
        Generate a utility score for each action. Utility is based on
        \begin{itemize}
            \item The character’s traits
            \item The relationship between the character and the target of the action (if
                applicable)
            \item How many story goals the action would progress
        \end{itemize}
    \item Pick an action from the pool randomly, where actions with higher utility scores
        are more likely to be chosen 
    \item Update the simulation state to account for the effects of the chosen action
\end{itemize}

Each action will have an associated natural-language representation template, which will
be filled in with the appropriate character names once the action is performed. This
natural-language string will then be written to the frontend to expose that event to the
reader as part of the simulation chronicle. 

\section{Analysis}
TBD

\section{Conclusion}
TBD

\subsection{Future Work}
TBD

\bibliography{refs}

\end{document}
